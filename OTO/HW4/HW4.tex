
\documentclass[a4paper]{article}

%Пакеты для математических символов:
\usepackage{amsmath} % американское математическое сообщество.
\usepackage{amssymb} % миллион разных значков и готический, ажурный шрифты.
\usepackage{amscd} % диаграммы, графики.
\usepackage{amsthm} % окружения теорем, определений и тд.
\usepackage{physics} % основные физические символы
%\usepackage{latexsym} % треугольники и пьяная стрелка.

%пакеты для шрифтов:
%\usepackage{euscript} % прописной шрифт с завитушками.
\usepackage{MnSymbol} % Значеки доказательства
\usepackage{verbatim} % улучшенный шрифт "пишущей машинки".
%\usepackage{array} % более удобные таблицы.
%\usepackage{multirow} % мультистолбцы в таблицах.
%\usepackage{longtable} % таблицы на несколько страниц.
%\usepackage{latexsym}

\usepackage{etoolbox}
\usepackage{slashbox} %Разделениени текста \backslashbox{}{}
\usepackage{collectbox} % Добавляет коробочки, можно складывать туда текст)


\usepackage{hyperref} % Ссылки как внешние так и внутренние
\hypersetup{
    colorlinks=true,
    linkcolor=black,
    filecolor=magenta,      
    urlcolor=cyan,
    pdftitle={Overleaf Example},
    pdfpagemode=FullScreen,
    }
    
%Пакеты для оформления:
\RequirePackage[center, medium]{titlesec}% Стиль секций и заголовков
%\usepackage[x11names]{xcolor} % 317 новых цветов для текста.
\usepackage{float} % Позволяет использовать H, h! для локации фигур
%\usepackage{multicol} % набор текста в несколько колонн.
\usepackage{graphicx} % расширенные возможности вставки стандартных картинок.
\usepackage{subcaption} % возможность вставлять картинки в строчку
%\usepackage{caption} % возможность подавить нумерацию у caption.
\usepackage{wrapfig} % вставка картинок и таблиц, обтекаемых текстом.
\usepackage{cancel} % значки для сокращения дробей, упрощения, стремления.
\usepackage{misccorr} % в заголовках появляется точка, но при ссылке на них ее нет.
%\usepackage{indentfirst} % отступ у первой строки раздела
%\usepackage{showkeys} % показывает label формул над их номером.
%\usepackage{fancyhdr} % удобное создание верхних и нижних колонтитулов.
%\usepackage{titlesec} % еще одно создание верхних и нижних колонтитулов
\usepackage{hyperref} % Ссылки как внешние так и внутренние
\hypersetup{
    colorlinks=true,
    linkcolor=black,
    filecolor=magenta,      
    urlcolor=cyan,
    pdftitle={Overleaf Example},
    pdfpagemode=FullScreen,
    }
\usepackage{xcolor} %Позволяет перекрасить все страници
\definecolor{mycolor}{RGB}{244,228,215} %Цвет перекраски


%Пакеты шрифтов, кодировок. НЕ МЕНЯТЬ РАСПОЛОЖЕНИЕ.
\usepackage[utf8]{inputenc} % кодировка символов.
%\usepackage{mathtext} % позволяет использовать русские буквы в формулах. НЕСОВМЕСТИМО С tempora.
\usepackage[T1, T2A]{fontenc} % кодировка шрифта.
\usepackage[english, russian]{babel} % доступные языки.


%Отступы и поля:
%размеры страницы А4 11.7x8.3in
\textwidth=7.3in % ширина текста
\textheight=10in % высота текста
\oddsidemargin=-0.5in % левый отступ(базовый 1дюйм + значение)
\topmargin=-0.5in % отступ сверху до колонтитула(базовый 1дюйм + значение)


%Сокращения
%Скобочки
\newcommand{\inrad}[1]{\left( #1 \right)}
\newcommand{\inner}[1]{\left( #1 \right)}
\newcommand{\infig}[1]{\left{ #1 \right}}
\newcommand{\insqr}[1]{\left[ #1 \right]}
\newcommand{\ave}[1]{\left\langle #1 \right\rangle}


\newcommand{\piv}[2]{\cfrac{\partial #1}{\partial #2}}

%% Красивые <= и >=
\renewcommand{\geq}{\geqslant}
\renewcommand{\leq}{\leqslant}

%%Значек выполнятся
\newcommand{\per}{\hookrightarrow}

%%Векторная алгебра
\newcommand{\rot}{\text{rot}}
\renewcommand{\div}{\text{div}}
\renewcommand{\grad}{\text{grad}}

%% Более привычные греческие буквы
\renewcommand{\phi}{\varphi}
\renewcommand{\epsilon}{\varepsilon}
\newcommand{\eps}{\varepsilon}
\newcommand{\com}{\mathbb{C}}
\newcommand{\re}{\mathbb{R}}
\newcommand{\nat}{\mathbb{N}}
\newcommand{\stp}{$\filledmedtriangleleft$}
\newcommand{\enp}{$\filledmedsquare$}
\newcommand{\crist}[3]{\cfrac{1}{2} \inner{g_{#1#2,#3} + g_{#1#3,#2} - g_{#2#3,#1}}}

\makeatletter
\newcommand{\sqbox}{%
    \collectbox{%
        \@tempdima=\dimexpr\width-\totalheight\relax
        \ifdim\@tempdima<\z@
            \fbox{\hbox{\hspace{-.5\@tempdima}\BOXCONTENT\hspace{-.5\@tempdima}}}%
        \else
            \ht\collectedbox=\dimexpr\ht\collectedbox+.5\@tempdima\relax
            \dp\collectedbox=\dimexpr\dp\collectedbox+.5\@tempdima\relax
            \fbox{\BOXCONTENT}%
        \fi
    }%
}
\makeatother
\newcommand{\mergelines}[2]{
\begin{tabular}{llp{.5\textwidth}}
#1 \\ #2
\end{tabular}
}
\newcommand\tab[1][0.51cm]{\hspace*{#1}}
\newcommand\difh[2]{\frac{\partial #1}{\partial #2}}
\newcommand{\messageforpeople}[1]{HSE Faculty of Physics \ \ HSE Faculty of Physics HSE Faculty of Physics \ \ HSE Faculty of Physics HSE Faculty of Physics \ \ HSE Faculty of Physics HSE Faculty of Physics \ \ HSE Faculty of Physics HSE Faculty of Physics \ \ HSE Faculty of Physics HSE Faculty of Physics \ \ HSE Faculty of Physics HSE Faculty of Physics \ \ HSE Faculty of Physics HSE Faculty of Physics \ \ HSE Faculty of Physics }


\numberwithin{equation}{section}

\begin{document}


\begin{flushright}
    Выполнил:
    Карибджанов Матвей

    Домашняя работа № 4
\end{flushright}
\tableofcontents
\newpagestyle{main}{
\setfootrule{0.4pt}
\setfoot{}{\thepage}{\sectiontitle \ № \thesection}}
\pagestyle{main}
% \pagecolor{mycolor}

\section{Задние}

Доказать что:
\begin{equation}
    A^i_{\ ;i} = \cfrac{1}{\sqrt{-g}} \piv{}{x^i}\inner{\sqrt{-g} A^i}
\end{equation}

Справа получим:

\begin{equation}
    \piv{}{x^i}\inner{\sqrt{-g} A^i} = 
    \piv{\sqrt{-g}}{x^i}A^i + \sqrt{-g}\piv{A^i}{x^i} = 
    -\cfrac{1}{2\sqrt{-g}} \piv{g}{x^i}A^i + \sqrt{-g}\piv{A^i}{x^i} =
    \cfrac{gg_{jk}g^{jk}_{\ \ ,i}}{2\sqrt{-g}} A^i + \sqrt{-g}\piv{A^i}{x^i}
\end{equation}

Слева распишем по определению:
\begin{equation}
    A^i_{\ ;i} = \piv{A^i}{x^i} + \Gamma^j_{\ lj} A^{l} 
\end{equation}


Седеним и сократим что можем:
\begin{equation}
    \piv{A^i}{x^i} + \Gamma^j_{\ lj} A^{l} = 
    \cfrac{1}{\cancel{\sqrt{-g}}} \cfrac{\cancel{g}g_{jk}g^{jk}_{\ \ ,i}}{2\cancel{\sqrt{-g}}} A^i 
    + \cfrac{\sqrt{-g}}{\cancel{\sqrt{-g}}} \piv{A^i}{x^i}
    =\cfrac{1}{2} g_{jk}g^{jk}_{\ \ ,i} A^i + \piv{A^i}{x^i}
\end{equation}

\begin{equation}
    \label{eq_1}
    \cancel{\piv{A^i}{x^i}} + \Gamma^j_{\ lj} A^{l} = 
    \cfrac{1}{2} g_{jk}g^{jk}_{\ \ ,i} A^i + \cancel{\piv{A^i}{x^i}}
\end{equation}

Теперь поиграемся:

\begin{equation}
    \Gamma_{ijk} + \Gamma_{jik} = \crist{i}{j}{k} + \crist{j}{i}{k} = g_{ij,k}
\end{equation}

\begin{equation}
    g^{ij}_{\ \ ,k} = -\inner{\Gamma_{lnk} + \Gamma_{nlk}} g^{nj} g^{li}
\end{equation}

\begin{equation}
    \label{eq_2}
    g_{jk}g^{jk}_{\ \ ,i} = -\inner{\Gamma_{lni} + \Gamma_{nli}} g^{nk} g^{lj }g_{jk} = -\inner{\Gamma_{lni} + \Gamma_{nli}} g^{nl} = 2 \Gamma^l_{\ li} = 2 \Gamma^l_{\ il}
\end{equation}

Подставляем в \ref{eq_1} и получем верное тождество.



\section{Задание}


Доказать что:
\begin{equation}
    \Gamma^{i}_{\ ij} = \ln \sqrt{-g}_{,j}
\end{equation}


Как всегда начнем с права:

\begin{equation}
    \ln\sqrt{-g}_{,l} = \cfrac{g_{,l}}{2g} = \cfrac{1}{2} g^{ij}g_{ij,l}
\end{equation}


Из предыдущей дз мы знаем чему равно $g_{'j}$, так же я уже считал 
$g^{ij}g_{ij,l}$ в \ref{eq_2}поэтому равентсво очевидно.


\end{document}