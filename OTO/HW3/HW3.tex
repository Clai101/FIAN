
\documentclass[a4paper]{article}

%Пакеты для математических символов:
\usepackage{amsmath} % американское математическое сообщество.
\usepackage{amssymb} % миллион разных значков и готический, ажурный шрифты.
\usepackage{amscd} % диаграммы, графики.
\usepackage{amsthm} % окружения теорем, определений и тд.
\usepackage{physics} % основные физические символы
%\usepackage{latexsym} % треугольники и пьяная стрелка.

%пакеты для шрифтов:
%\usepackage{euscript} % прописной шрифт с завитушками.
\usepackage{MnSymbol} % Значеки доказательства
\usepackage{verbatim} % улучшенный шрифт "пишущей машинки".
%\usepackage{array} % более удобные таблицы.
%\usepackage{multirow} % мультистолбцы в таблицах.
%\usepackage{longtable} % таблицы на несколько страниц.
%\usepackage{latexsym}

\usepackage{etoolbox}
\usepackage{slashbox} %Разделениени текста \backslashbox{}{}
\usepackage{collectbox} % Добавляет коробочки, можно складывать туда текст)


\usepackage{hyperref} % Ссылки как внешние так и внутренние
\hypersetup{
    colorlinks=true,
    linkcolor=black,
    filecolor=magenta,      
    urlcolor=cyan,
    pdftitle={Overleaf Example},
    pdfpagemode=FullScreen,
    }
    
%Пакеты для оформления:
\RequirePackage[center, medium]{titlesec}% Стиль секций и заголовков
%\usepackage[x11names]{xcolor} % 317 новых цветов для текста.
\usepackage{float} % Позволяет использовать H, h! для локации фигур
%\usepackage{multicol} % набор текста в несколько колонн.
\usepackage{graphicx} % расширенные возможности вставки стандартных картинок.
\usepackage{subcaption} % возможность вставлять картинки в строчку
%\usepackage{caption} % возможность подавить нумерацию у caption.
\usepackage{wrapfig} % вставка картинок и таблиц, обтекаемых текстом.
\usepackage{cancel} % значки для сокращения дробей, упрощения, стремления.
\usepackage{misccorr} % в заголовках появляется точка, но при ссылке на них ее нет.
%\usepackage{indentfirst} % отступ у первой строки раздела
%\usepackage{showkeys} % показывает label формул над их номером.
%\usepackage{fancyhdr} % удобное создание верхних и нижних колонтитулов.
%\usepackage{titlesec} % еще одно создание верхних и нижних колонтитулов
\usepackage{hyperref} % Ссылки как внешние так и внутренние
\hypersetup{
    colorlinks=true,
    linkcolor=black,
    filecolor=magenta,      
    urlcolor=cyan,
    pdftitle={Overleaf Example},
    pdfpagemode=FullScreen,
    }
\usepackage{xcolor} %Позволяет перекрасить все страници
\definecolor{mycolor}{RGB}{244,228,215} %Цвет перекраски


%Пакеты шрифтов, кодировок. НЕ МЕНЯТЬ РАСПОЛОЖЕНИЕ.
\usepackage[utf8]{inputenc} % кодировка символов.
%\usepackage{mathtext} % позволяет использовать русские буквы в формулах. НЕСОВМЕСТИМО С tempora.
\usepackage[T1, T2A]{fontenc} % кодировка шрифта.
\usepackage[english, russian]{babel} % доступные языки.


%Отступы и поля:
%размеры страницы А4 11.7x8.3in
\textwidth=7.3in % ширина текста
\textheight=10in % высота текста
\oddsidemargin=-0.5in % левый отступ(базовый 1дюйм + значение)
\topmargin=-0.5in % отступ сверху до колонтитула(базовый 1дюйм + значение)


%Сокращения
%Скобочки
\newcommand{\inrad}[1]{\left( #1 \right)}
\newcommand{\inner}[1]{\left( #1 \right)}
\newcommand{\infig}[1]{\left{ #1 \right}}
\newcommand{\insqr}[1]{\left[ #1 \right]}
\newcommand{\ave}[1]{\left\langle #1 \right\rangle}


\newcommand{\piv}[2]{\cfrac{\partial #1}{\partial #2}}

%% Красивые <= и >=
\renewcommand{\geq}{\geqslant}
\renewcommand{\leq}{\leqslant}

%%Значек выполнятся
\newcommand{\per}{\hookrightarrow}

%%Векторная алгебра
\newcommand{\rot}{\text{rot}}
\renewcommand{\div}{\text{div}}
\renewcommand{\grad}{\text{grad}}

%% Более привычные греческие буквы
\renewcommand{\phi}{\varphi}
\renewcommand{\epsilon}{\varepsilon}
\newcommand{\eps}{\varepsilon}
\newcommand{\com}{\mathbb{C}}
\newcommand{\re}{\mathbb{R}}
\newcommand{\nat}{\mathbb{N}}
\newcommand{\stp}{$\filledmedtriangleleft$}
\newcommand{\enp}{$\filledmedsquare$}

\makeatletter
\newcommand{\sqbox}{%
    \collectbox{%
        \@tempdima=\dimexpr\width-\totalheight\relax
        \ifdim\@tempdima<\z@
            \fbox{\hbox{\hspace{-.5\@tempdima}\BOXCONTENT\hspace{-.5\@tempdima}}}%
        \else
            \ht\collectedbox=\dimexpr\ht\collectedbox+.5\@tempdima\relax
            \dp\collectedbox=\dimexpr\dp\collectedbox+.5\@tempdima\relax
            \fbox{\BOXCONTENT}%
        \fi
    }%
}
\makeatother
\newcommand{\mergelines}[2]{
\begin{tabular}{llp{.5\textwidth}}
#1 \\ #2
\end{tabular}
}
\newcommand\tab[1][0.51cm]{\hspace*{#1}}
\newcommand\difh[2]{\frac{\partial #1}{\partial #2}}
\newcommand{\messageforpeople}[1]{HSE Faculty of Physics \ \ HSE Faculty of Physics HSE Faculty of Physics \ \ HSE Faculty of Physics HSE Faculty of Physics \ \ HSE Faculty of Physics HSE Faculty of Physics \ \ HSE Faculty of Physics HSE Faculty of Physics \ \ HSE Faculty of Physics HSE Faculty of Physics \ \ HSE Faculty of Physics HSE Faculty of Physics \ \ HSE Faculty of Physics HSE Faculty of Physics \ \ HSE Faculty of Physics }


\numberwithin{equation}{section}

\begin{document}


\begin{flushright}
    Выполнил:
    Карибджанов Матвей

    Домашнее задание № 3
\end{flushright}
\newpagestyle{main}{
\setfootrule{0.4pt}
\setfoot{}{\thepage}{Домашнее задание № 3}}
\pagestyle{main}
% \pagecolor{mycolor}

Нати все не нулевые компоненты тензора римана в метрике 
$ds^2 = d\theta^2 + \sin^2 \theta d\phi^2$.
\begin{gather}
    g_{ij} = 
    \begin{pmatrix}
        1 & 0 \\
        0 & \sin^2 \theta
    \end{pmatrix}
\end{gather}

По свойствам тензора римана в 2 мерье у него 
всего одна независимая компонента поэтому давате найдем:

\begin{equation}
    \label{eq_riman}
    R^i_{\ jij} = \Gamma^{i}_{\ jj,i} - \Gamma^{i}_{\ ij,j} 
    + \Gamma^{i}_{\ il}\Gamma^{l}_{\ jj} 
    - \Gamma^{i}_{\ jl}\Gamma^{l}_{\ ij}
\end{equation}

Здесь не подразумеватся свертка по индексам ij а подчеркиватся свойство 
симметричноати по паре ниндксов, что в дальнешем может упротить решение,
позволив занутить или тп каой-то член. Восползуемся:

\begin{equation}
    \partial_l g_{ij} = g_{mj} \Gamma^m_{\ il} + g_{im} \Gamma^m_{\ jl} 
    = g_{jm} \Gamma^m_{\ il} + g_{im} \Gamma^m_{\ jl}
\end{equation}

Перепишем это выражение

\begin{gather}
    \label{eq_l1}
    l = 1: 
    \begin{pmatrix}
        0 & 0 \\
        0 & 2 \sin \theta \cos \theta
    \end{pmatrix}
    = 
    \begin{pmatrix}
        \Gamma^1_{\ 11} & \Gamma^1_{\ 21} \\
        \sin^2 \theta \Gamma^2_{\ 11} & \sin^2 \theta \Gamma^2_{\ 21}
    \end{pmatrix}
    + 
    \begin{pmatrix}
        \Gamma^1_{\ 11} & \sin^2 \theta \Gamma^2_{\ 11}  \\
        \Gamma^1_{\ 21} & \sin^2 \theta \Gamma^2_{\ 21}
    \end{pmatrix}
\end{gather}

Видим что $\Gamma^1_{\ 11} = 0$ 

\begin{gather}
    \label{eq_l2}
    l = 2: 
    \begin{pmatrix}
        0 & 0 \\
        0 & 0
    \end{pmatrix}
    = 
    \begin{pmatrix}
        \Gamma^1_{\ 12} & \Gamma^1_{\ 22} \\
        \sin^2 \theta \Gamma^2_{\ 12} & \sin^2 \theta \Gamma^2_{\ 22}
    \end{pmatrix}
    + 
    \begin{pmatrix}
        \Gamma^1_{\ 12} & \sin^2 \theta \Gamma^2_{\ 12}  \\
        \Gamma^1_{\ 22} & \sin^2 \theta \Gamma^2_{\ 22}
    \end{pmatrix}
\end{gather}

Видим что $\Gamma^2_{\ 22} = \Gamma^1_{\ 12} = \Gamma^1_{\ 21} = 0$ 
и подставим в \ref{eq_l1} получим что $\Gamma^2_{\ 11} = 0$. 
Потому будем  считать компонеты $\Gamma^{2}_{\ 21}, \Gamma^{2}_{\ 12}, \Gamma^{1}_{\ 22}$

Из \ref{eq_l1} следует:
\begin{gather}
    2 \sin \theta \cos \theta = 2 \sin^2 \theta \Gamma^2_{\ 21} \implies 
    \Gamma^2_{\ 21} = \Gamma^2_{\ 12} = \cot \theta
\end{gather}

Из \ref{eq_l2} следует:
\begin{equation}
    -\sin^2 \theta \Gamma^2_{\ 12} = \Gamma^1_{\ 22} \implies 
    \Gamma^1_{\ 22} = - \sin \theta \cos \theta
\end{equation}

В итоге подставм в \ref{eq_riman}:

\begin{equation}
    R^{1}_{\ 212} = - \partial_\theta \cos\theta \sin\theta + 0 + 0
    + \cos\theta \sin\theta \cot \theta = \sin^2 \theta
\end{equation}



\end{document}