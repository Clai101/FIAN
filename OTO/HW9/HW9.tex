
\documentclass[a4paper]{article}

%Пакеты для математических символов:
\usepackage{amsmath} % американское математическое сообщество.
\usepackage{amssymb} % миллион разных значков и готический, ажурный шрифты.
\usepackage{amscd} % диаграммы, графики.
\usepackage{amsthm} % окружения теорем, определений и тд.
\usepackage{physics} % основные физические символы
%\usepackage{latexsym} % треугольники и пьяная стрелка.

%пакеты для шрифтов:
%\usepackage{euscript} % прописной шрифт с завитушками.
\usepackage{MnSymbol} % Значеки доказательства
\usepackage{verbatim} % улучшенный шрифт "пишущей машинки".
%\usepackage{array} % более удобные таблицы.
%\usepackage{multirow} % мультистолбцы в таблицах.
%\usepackage{longtable} % таблицы на несколько страниц.
%\usepackage{latexsym}

\usepackage{etoolbox}
\usepackage{slashbox} %Разделениени текста \backslashbox{}{}
\usepackage{collectbox} % Добавляет коробочки, можно складывать туда текст)


\usepackage{hyperref} % Ссылки как внешние так и внутренние
\hypersetup{
    colorlinks=true,
    linkcolor=black,
    filecolor=magenta,      
    urlcolor=cyan,
    pdftitle={Overleaf Example},
    pdfpagemode=FullScreen,
    }
    
%Пакеты для оформления:
\RequirePackage[center, medium]{titlesec}% Стиль секций и заголовков
%\usepackage[x11names]{xcolor} % 317 новых цветов для текста.
\usepackage{float} % Позволяет использовать H, h! для локации фигур
%\usepackage{multicol} % набор текста в несколько колонн.
\usepackage{graphicx} % расширенные возможности вставки стандартных картинок.
\usepackage{subcaption} % возможность вставлять картинки в строчку
%\usepackage{caption} % возможность подавить нумерацию у caption.
\usepackage{wrapfig} % вставка картинок и таблиц, обтекаемых текстом.
\usepackage{cancel} % значки для сокращения дробей, упрощения, стремления.
\usepackage{misccorr} % в заголовках появляется точка, но при ссылке на них ее нет.
%\usepackage{indentfirst} % отступ у первой строки раздела
%\usepackage{showkeys} % показывает label формул над их номером.
%\usepackage{fancyhdr} % удобное создание верхних и нижних колонтитулов.
%\usepackage{titlesec} % еще одно создание верхних и нижних колонтитулов
\usepackage{hyperref} % Ссылки как внешние так и внутренние
\hypersetup{
    colorlinks=true,
    linkcolor=black,
    filecolor=magenta,      
    urlcolor=cyan,
    pdftitle={Overleaf Example},
    pdfpagemode=FullScreen,
    }
\usepackage{xcolor} %Позволяет перекрасить все страници
\definecolor{mycolor}{RGB}{244,228,215} %Цвет перекраски


%Пакеты шрифтов, кодировок. НЕ МЕНЯТЬ РАСПОЛОЖЕНИЕ.
\usepackage[utf8]{inputenc} % кодировка символов.
%\usepackage{mathtext} % позволяет использовать русские буквы в формулах. НЕСОВМЕСТИМО С tempora.
\usepackage[T1, T2A]{fontenc} % кодировка шрифта.
\usepackage[english, russian]{babel} % доступные языки.


%Отступы и поля:
%размеры страницы А4 11.7x8.3in
\textwidth=7.3in % ширина текста
\textheight=10in % высота текста
\oddsidemargin=-0.5in % левый отступ(базовый 1дюйм + значение)
\topmargin=-0.5in % отступ сверху до колонтитула(базовый 1дюйм + значение)


%Сокращения
%Скобочки
\newcommand{\inrad}[1]{\left( #1 \right)}
\newcommand{\inner}[1]{\left( #1 \right)}
\newcommand{\infig}[1]{\left{ #1 \right}}
\newcommand{\insqr}[1]{\left[ #1 \right]}
\newcommand{\ave}[1]{\left\langle #1 \right\rangle}



%% Красивые <= и >=
\renewcommand{\geq}{\geqslant}
\renewcommand{\leq}{\leqslant}

%%Значек выполнятся
\newcommand{\per}{\hookrightarrow}

%%Векторная алгебра
\newcommand{\rot}{\text{rot}}
\renewcommand{\div}{\text{div}}
\renewcommand{\grad}{\text{grad}}

%% Более привычные греческие буквы
\renewcommand{\phi}{\varphi}
\renewcommand{\epsilon}{\varepsilon}
\newcommand{\eps}{\varepsilon}
\newcommand{\com}{\mathbb{C}}
\newcommand{\re}{\mathbb{R}}
\newcommand{\nat}{\mathbb{N}}
\newcommand{\stp}{$\filledmedtriangleleft$}
\newcommand{\enp}{$\filledmedsquare$}

%%Тензорный анализ ОТО теория поля
\newcommand{\Li}[1]{\mathfrak{L}_{#1}}
\newcommand{\crist}[3]{\cfrac{1}{2} \inner{g_{#1#2,#3} + g_{#1#3,#2} - g_{#2#3,#1}}}
\newcommand{\piv}[2]{\cfrac{\partial #1}{\partial #2}}

\makeatletter
\newcommand{\sqbox}{%
    \collectbox{%
        \@tempdima=\dimexpr\width-\totalheight\relax
        \ifdim\@tempdima<\z@
            \fbox{\hbox{\hspace{-.5\@tempdima}\BOXCONTENT\hspace{-.5\@tempdima}}}%
        \else
            \ht\collectedbox=\dimexpr\ht\collectedbox+.5\@tempdima\relax
            \dp\collectedbox=\dimexpr\dp\collectedbox+.5\@tempdima\relax
            \fbox{\BOXCONTENT}%
        \fi
    }%
}
\makeatother
\newcommand{\mergelines}[2]{
\begin{tabular}{llp{.5\textwidth}}
#1 \\ #2
\end{tabular}
}
\newcommand\tab[1][0.51cm]{\hspace*{#1}}
\newcommand\difh[2]{\frac{\partial #1}{\partial #2}}
\newcommand{\messageforpeople}[1]{HSE Faculty of Physics \ \ HSE Faculty of Physics HSE Faculty of Physics \ \ HSE Faculty of Physics HSE Faculty of Physics \ \ HSE Faculty of Physics HSE Faculty of Physics \ \ HSE Faculty of Physics HSE Faculty of Physics \ \ HSE Faculty of Physics HSE Faculty of Physics \ \ HSE Faculty of Physics HSE Faculty of Physics \ \ HSE Faculty of Physics HSE Faculty of Physics \ \ HSE Faculty of Physics }


\numberwithin{equation}{section}

\begin{document}


\begin{flushright}
    Выполнил:
    Карибджанов Матвей

    Домашняя работа № 9
\end{flushright}
\newpagestyle{main}{
\setfootrule{0.4pt}
\setfoot{}{\thepage}{Домашняя работа \ № \ 9}}
\pagestyle{main}
% \pagecolor{mycolor}

Вывести уравнения описывающие теорию Бранса — Дикке, которая задаются действием:

\begin{eqnarray}
    S = \int_\Omega d^4 x \sqrt{-g} \inner{\phi R - \omega \cfrac{\phi_{,i}\phi^{,i}}{\phi}} 
\end{eqnarray}

\begin{eqnarray}
    \label{eq:var}
    \delta S = \int_\Omega d^4 x \inner{\phi R - \omega \cfrac{\phi_{,i}\phi^{,i}}{\phi}}  \delta \inner{-g}^{1/2}
    + \inner{-g}^{1/2} \delta \phi R 
    - \inner{-g}^{1/2} \delta\omega \cfrac{\phi_{,i}\phi^{,i}}{\phi}
\end{eqnarray}

Варьирование по метрике:

\begin{equation}
    \label{eq:d_phi_R}
    \delta \phi R = \phi \delta R = \phi \delta R_{ij} g^{ij} 
    = \phi g^{ij} \delta R_{ij}  + \phi R_{ij} \delta g^{ij}
\end{equation}

\begin{equation}
    \label{eq:phi_phi_g}
    \delta\omega \cfrac{\phi_{,i}\phi^{,i}}{\phi} 
    = \omega \cfrac{\phi^{,i}\phi^{,j}}{\phi} \delta g_{ij} \inner{-g}^{-1/2}
    = \omega \inner{-g}^{1/2} \cfrac{\phi^{,i}\phi^{,j}}{\phi} \insqr{ \delta g_{ij} 
    - \cfrac{1}{2} g_{ij} g^{kl} \delta g_{kl}}
\end{equation}

Для приведения всего к одному виду в \ref{eq:phi_phi_g} надо раскрыть скобки 
перименовть немые индексы во втором слагаемом.

\begin{equation}
    \cfrac{\phi^{,i}\phi^{,j}}{\phi} \insqr{\inner{-g}^{1/2} \delta g_{ij} 
    - \cfrac{1}{2} \inner{-g}^{1/2} g_{ij} g^{kl} \delta g_{kl}} 
    = \cfrac{\phi^{,i}\phi^{,j}}{\phi} \inner{-g}^{1/2} \delta g_{ij} 
    - \cfrac{1}{2} \inner{-g}^{1/2} \cfrac{\phi^{,l}\phi_{,l}}{\phi} g^{ij} \delta g_{ij}
\end{equation}

Упростим \ref{eq:d_phi_R}, для первого слагаемого можно 
сразу домножить на $\inner{-g}^{1/2}$ тк этот множитель стоит перед \ref{eq:d_phi_R} в \ref{eq:var}:

\begin{equation}
    \inner{-g}^{1/2} \phi g^{ij} \delta R_{ij} = 
    \phi\inner{g^{jk} \delta \Gamma^{i}_{jk} - g^{ji} \delta \Gamma^{k}_{jk}}_{,i}
    = 
    \cancel{\inner{ \phi g^{jk} \delta \Gamma^{i}_{jk} - \phi g^{ji} \delta \Gamma^{k}_{jk}}_{,i}}
    - \phi_{,i} \inner{g^{jk} \delta \Gamma^{i}_{jk} - g^{ji} \delta \Gamma^{k}_{jk}}
\end{equation}

Теперь варьируем $\Gamma^i_{jk}$ она у нас была на лекции, в принципе как и $R_{ij}$

\begin{equation}
    \delta \Gamma^i_{jk} = \cfrac{1}{2} g^{il} \insqr{\inner{\delta g_{lj}}_{;k} +
    \inner{\delta g_{lk}}_{;j} - \inner{\delta g_{jk}}_{;l}}
\end{equation}


Для удобства заметим, что если по  теорме Гаусса-Остроградского 
интеграл занулился для обычной производной, то он также занулится и для ковариантной 
производной, так как мы можем прейди в СО где $\nabla_i \to \partial_i$ посчитать интеграл и вернуться, 
так же вспомним $g_{ij;k} = 0$:

\begin{equation}
    \label{eq:1}
    - \phi_{,i} g^{jk} \delta \Gamma^{i}_{jk}
    = - \cfrac{1}{2} \phi_{,i} g^{jk} g^{il} \insqr{\inner{\delta g_{lj}}_{;k} +
    \inner{\delta g_{lk}}_{;j} - \inner{\delta g_{jk}}_{;l}}
    = \cfrac{1}{2} g^{jk} g^{il} \insqr{\phi_{,i;k}\delta g_{lj} +
    \phi_{,i;j} \delta g_{lk} - \phi_{,i;j} \delta g_{jk}}
\end{equation}

\begin{equation}
    \label{eq:2}
    \phi_{,i} \phi g^{ji} \delta \Gamma^{k}_{jk} =
    \cfrac{1}{2} \phi_{,i} g^{ji} g^{kl} \insqr{\inner{\delta g_{lj}}_{;k} +
    \inner{\delta g_{lk}}_{;j} - \inner{\delta g_{jk}}_{;l}}
    = - \cfrac{1}{2} g^{ji} g^{kl} \insqr{\phi_{,i;k} \delta g_{lj} +
    \phi_{,i;j}\delta g_{lk} - \phi_{,i}\delta g_{jk}}
\end{equation}

Поднятие индексоа и переиминовывание покажут, равентво перого члена из \ref{eq:2} с последниего из \ref{eq:1}, 
а налогично для перого из \ref{eq:1} и последнего \ref{eq:2}, а среднии члены
в свою очередь взаимосократятся, поэтому получим: 

\begin{equation}
    \inner{-g}^{1/2} \phi g^{ij} \phi \delta R_{ij} \Rightarrow 
    \insqr{-\phi_{,i;j} + g_{ij} \phi_{,l}^{;l}} \inner{-g}^{1/2} \delta g^{ij} 
\end{equation}

Варирование втрого члена из \ref{eq:phi_phi_g}, я проводить не буду, так как результат известен с леции, 
там получится тензор Эйнштейна. Поэтому вспоминим о вкладе материи в действие:

\begin{equation}
    16 \pi \cfrac{\delta \mathcal{L}_m}{\delta g_{ij}} = -8\pi T^{ij}
\end{equation}

Тогда просумировав :

\begin{equation}
    8\pi T_{ij} = \omega \cfrac{\phi_{,i}\phi_{,j} 
    - \cfrac{1}{2}g_{ij}\phi_{,i}\phi^{,i}}{\phi} - G_{ij}\phi
\end{equation}

Тепрь рассмтрим вриацию по полю, первый член из \ref{eq:var} занулится:

\begin{equation}
    \inner{-g}^{1/2} \delta \phi R  = \inner{-g}^{1/2} R \delta \phi
\end{equation}

\begin{equation}
    \inner{-g}^{1/2} \delta\omega \cfrac{\phi_{,i}\phi^{,i}}{\phi} =
    \inner{-g}^{1/2} \omega \phi_{,i}\phi^{,i} \delta \cfrac{1}{\phi} 
    + \inner{-g}^{1/2} \omega \cfrac{1}{\phi}  \delta \phi_{,i}\phi^{,i} 
    = -\inner{-g}^{1/2} \omega \phi_{,i}\phi^{,i} \cfrac{1}{\phi^2} \delta \phi 
    + 2  \omega \partial_i \inner{\cfrac{\phi_{,i}\inner{-g}^{1/2}}{\phi}} \delta \phi
\end{equation}

Для второго слагаемого снова воспользовался к теоремой Гаусса-Остроградского, а коэф.
2 появился из-аз двух одинаковых $\phi^{,i}$ и $\phi_{,i}$. Материя не дает вклад в 
варицию по полю, поэтому в итоге вариация по полю даст нам:

\begin{equation}
    \inner{-g}^{1/2} \inner{\omega \cfrac{\phi_{,i}\phi^{,i}}{\phi^2} + R} 
    + 2  \omega \partial_i \cfrac{\phi_{,i}\inner{-g}^{1/2}}{\phi} = 0
\end{equation}
\end{document}