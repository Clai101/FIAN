
\documentclass[a4paper]{article}

%Пакеты для математических символов:
\usepackage{amsmath} % американское математическое сообщество.
\usepackage{amssymb} % миллион разных значков и готический, ажурный шрифты.
\usepackage{amscd} % диаграммы, графики.
\usepackage{amsthm} % окружения теорем, определений и тд.
\usepackage{physics} % основные физические символы
%\usepackage{latexsym} % треугольники и пьяная стрелка.

%пакеты для шрифтов:
%\usepackage{euscript} % прописной шрифт с завитушками.
\usepackage{MnSymbol} % Значеки доказательства
\usepackage{verbatim} % улучшенный шрифт "пишущей машинки".
%\usepackage{array} % более удобные таблицы.
%\usepackage{multirow} % мультистолбцы в таблицах.
%\usepackage{longtable} % таблицы на несколько страниц.
%\usepackage{latexsym}

\usepackage{etoolbox}
\usepackage{slashbox} %Разделениени текста \backslashbox{}{}
\usepackage{collectbox} % Добавляет коробочки, можно складывать туда текст)


\usepackage{hyperref} % Ссылки как внешние так и внутренние
\hypersetup{
    colorlinks=true,
    linkcolor=black,
    filecolor=magenta,      
    urlcolor=cyan,
    pdftitle={Overleaf Example},
    pdfpagemode=FullScreen,
    }
    
%Пакеты для оформления:
\RequirePackage[center, medium]{titlesec}% Стиль секций и заголовков
%\usepackage[x11names]{xcolor} % 317 новых цветов для текста.
\usepackage{float} % Позволяет использовать H, h! для локации фигур
%\usepackage{multicol} % набор текста в несколько колонн.
\usepackage{graphicx} % расширенные возможности вставки стандартных картинок.
\usepackage{subcaption} % возможность вставлять картинки в строчку
%\usepackage{caption} % возможность подавить нумерацию у caption.
\usepackage{wrapfig} % вставка картинок и таблиц, обтекаемых текстом.
\usepackage{cancel} % значки для сокращения дробей, упрощения, стремления.
\usepackage{misccorr} % в заголовках появляется точка, но при ссылке на них ее нет.
%\usepackage{indentfirst} % отступ у первой строки раздела
%\usepackage{showkeys} % показывает label формул над их номером.
%\usepackage{fancyhdr} % удобное создание верхних и нижних колонтитулов.
%\usepackage{titlesec} % еще одно создание верхних и нижних колонтитулов
\usepackage{hyperref} % Ссылки как внешние так и внутренние
\hypersetup{
    colorlinks=true,
    linkcolor=black,
    filecolor=magenta,      
    urlcolor=cyan,
    pdftitle={Overleaf Example},
    pdfpagemode=FullScreen,
    }
\usepackage{xcolor} %Позволяет перекрасить все страници
\definecolor{mycolor}{RGB}{244,228,215} %Цвет перекраски


%Пакеты шрифтов, кодировок. НЕ МЕНЯТЬ РАСПОЛОЖЕНИЕ.
\usepackage[utf8]{inputenc} % кодировка символов.
%\usepackage{mathtext} % позволяет использовать русские буквы в формулах. НЕСОВМЕСТИМО С tempora.
\usepackage[T1, T2A]{fontenc} % кодировка шрифта.
\usepackage[english, russian]{babel} % доступные языки.


%Отступы и поля:
%размеры страницы А4 11.7x8.3in
\textwidth=7.3in % ширина текста
\textheight=10in % высота текста
\oddsidemargin=-0.5in % левый отступ(базовый 1дюйм + значение)
\topmargin=-0.5in % отступ сверху до колонтитула(базовый 1дюйм + значение)


%Сокращения
%Скобочки
\newcommand{\inrad}[1]{\left( #1 \right)}
\newcommand{\inner}[1]{\left( #1 \right)}
\newcommand{\infig}[1]{\left{ #1 \right}}
\newcommand{\insqr}[1]{\left[ #1 \right]}
\newcommand{\ave}[1]{\left\langle #1 \right\rangle}



%% Красивые <= и >=
\renewcommand{\geq}{\geqslant}
\renewcommand{\leq}{\leqslant}

%%Значек выполнятся
\newcommand{\per}{\hookrightarrow}

%%Векторная алгебра
\newcommand{\rot}{\text{rot}}
\renewcommand{\div}{\text{div}}
\renewcommand{\grad}{\text{grad}}

%% Более привычные греческие буквы
\renewcommand{\phi}{\varphi}
\renewcommand{\epsilon}{\varepsilon}
\newcommand{\eps}{\varepsilon}
\newcommand{\com}{\mathbb{C}}
\newcommand{\re}{\mathbb{R}}
\newcommand{\nat}{\mathbb{N}}
\newcommand{\stp}{$\filledmedtriangleleft$}
\newcommand{\enp}{$\filledmedsquare$}

%%Тензорный анализ ОТО теория поля
\newcommand{\Li}[1]{\mathfrak{L}_{#1}}
\newcommand{\crist}[3]{\cfrac{1}{2} \inner{g_{#1#2,#3} + g_{#1#3,#2} - g_{#2#3,#1}}}
\newcommand{\piv}[2]{\cfrac{\partial #1}{\partial #2}}

\makeatletter
\newcommand{\sqbox}{%
    \collectbox{%
        \@tempdima=\dimexpr\width-\totalheight\relax
        \ifdim\@tempdima<\z@
            \fbox{\hbox{\hspace{-.5\@tempdima}\BOXCONTENT\hspace{-.5\@tempdima}}}%
        \else
            \ht\collectedbox=\dimexpr\ht\collectedbox+.5\@tempdima\relax
            \dp\collectedbox=\dimexpr\dp\collectedbox+.5\@tempdima\relax
            \fbox{\BOXCONTENT}%
        \fi
    }%
}
\makeatother
\newcommand{\mergelines}[2]{
\begin{tabular}{llp{.5\textwidth}}
#1 \\ #2
\end{tabular}
}
\newcommand\tab[1][0.51cm]{\hspace*{#1}}
\newcommand\difh[2]{\frac{\partial #1}{\partial #2}}
\newcommand{\messageforpeople}[1]{HSE Faculty of Physics \ \ HSE Faculty of Physics HSE Faculty of Physics \ \ HSE Faculty of Physics HSE Faculty of Physics \ \ HSE Faculty of Physics HSE Faculty of Physics \ \ HSE Faculty of Physics HSE Faculty of Physics \ \ HSE Faculty of Physics HSE Faculty of Physics \ \ HSE Faculty of Physics HSE Faculty of Physics \ \ HSE Faculty of Physics HSE Faculty of Physics \ \ HSE Faculty of Physics }


\numberwithin{equation}{section}

\begin{document}


\begin{flushright}
    Выполнил:
    Карибджанов Матвей

    Домашняя работа № 6
\end{flushright}
\newpagestyle{main}{
\setfootrule{0.4pt}
\setfoot{}{\thepage}{Домашняя работа \ № \ 7}}
\pagestyle{main}
% \pagecolor{mycolor}

Надем вектора Киллинага в системе с метрикой:
\begin{gather}
    g_{ij} = 
    \begin{pmatrix}
        \sin^2 \theta & 0 \\
        0 & 1
    \end{pmatrix}
\end{gather}

Из дз 3 мы уже знаем что в такой системе не нелулевыми 
будут только компонетами афинной связности будут 
$\Gamma^{2}_{\ 21}, \Gamma^{2}_{\ 12}, \Gamma^{1}_{\ 22}$, 
они имеют следующие значения:

\begin{equation}
    \Gamma^{2}_{\ 21} = \Gamma^{2}_{\ 12} = \cot \theta, \ \Gamma^{2}_{\ 11} = - \sin \theta \cos \theta
\end{equation}

Теперь подставим в формулу

\begin{equation}
    \xi_{i;j} + \xi_{i;j} = \xi_{i,j} - \xi_{k} \Gamma^{k}_{\ ij} 
    + \xi_{j,i} - \xi_{k} \Gamma^{k}_{\ ji} = 
    \xi_{i,j} + \xi_{j,i} - 2 \xi_{k} \Gamma^{k}_{\ ji}
\end{equation}

\begin{gather}
    \begin{pmatrix}
        \piv{\xi_1}{\phi} & \piv{\xi_1}{\theta} \\
        \piv{\xi_2}{\phi} & \piv{\xi_2}{\theta}
    \end{pmatrix}
    + 
    \begin{pmatrix}
        \piv{\xi_1}{\phi} & \piv{\xi_2}{\phi} \\
        \piv{\xi_1}{\theta} & \piv{\xi_2}{\theta}
    \end{pmatrix}
    - 
    2\xi_2
    \begin{pmatrix}
        -\sin \theta \cos \theta & \cot \theta\\
        \cot \theta & 0
    \end{pmatrix}
\end{gather}

Перкпишем в виде:

Из из элемента мариц 22 поймем что:

\begin{equation}
    \xi_2 = \xi_2(\phi) 
\end{equation}

Подставим в 11:

\begin{equation}
    \xi_1 = -2 \sin \theta \cos \theta \int \xi_2 d\phi +C(\theta)
\end{equation}

А теперь посмтрим что в побочной диаганали:

\begin{equation}
    \xi_2' + \inner{\cos^2\theta - \sin^2\theta} \int \xi_2 d\phi + C'(\theta) = 
    2 C(\theta)cot(\theta) - 2 \cos^2 \theta \int \xi_2 d \phi
\end{equation}

\begin{equation}
    \xi_2' + \int \xi_2 d\phi = - C'(\theta) + 2 C(\theta) \cot \theta
\end{equation}

Можем разбить на 2 уравнения:

\begin{eqnarray}
    \xi_2' + \int \xi_2 d\phi = b\\
    C'(\theta) - 2 C(\theta) \cot \theta = -b
\end{eqnarray}

\begin{equation}
    \label{eq:1}
     \xi_2 = d \cos \phi  + a \sin \phi   
\end{equation}

\begin{equation}
    C(\theta) = \inner{b \cot \theta + f} \sin^2 \theta
\end{equation}

Так как для решения \ref{eq:1} я дифференцировал то теряю ин формацию 
об одной костанте поэтому надо подтавить и востановить:

\begin{equation}
    -d \sin \phi + a \cos \phi + d \sin \phi - a \cos\phi = b \implies b = 0
\end{equation}

В итоге получим:

\begin{eqnarray}
    C(\theta) = f \sin^2 \theta \\
    \xi_2 = d \cos \phi  + a \sin \phi
\end{eqnarray}

\begin{gather}
    \xi_i = 
    \begin{pmatrix}
        f \cos^2 \theta - \sin \theta \cos \theta \inner{d \sin \phi - a \cos \phi}\\
        d \cos \phi + a \sin \phi
    \end{pmatrix}
\end{gather}




\end{document}